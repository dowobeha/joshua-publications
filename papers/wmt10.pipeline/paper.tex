\documentclass[11pt]{article}
\usepackage{times}
\usepackage{url}
\usepackage{latexsym}
\usepackage[%
  colorlinks,%
  linkcolor=black,%
  anchorcolor=black,%
  citecolor=black,%
  filecolor=black,%
  menucolor=black,%
  pagecolor=black,%
  urlcolor=black,%
  %hyperindex=false,%
  bookmarksopen=true,%
  pdfusetitle,%
  pdfpagelabels,
  naturalnames
]{hyperref}
%\usepackage[colorlinks]{hyperref}


\usepackage{acl2010}
\usepackage[colorinlistoftodos]{todonotes}


\title{Empirical Questions in Machine Translation\
 --- Reproducible Results for the JHU WMT10 System}

\author{Lane Schwartz\
\thanks{Research conducted as a visiting researcher at Johns Hopkins University}\
\\University of Minnesota\\Minneapolis, MN}

\date{}

\begin{document}
\maketitle

\begin{abstract}
\end{abstract}

%\listoftodos

\section{Introduction}



\todo[inline,caption={Motivation}]{
Papers are not reproducible.
\begin{itemize}
\item Don't list code
\item Don't provide data
\item Don't publish scripts to reproduce
\end{itemize}

MT is complex, much more so than other CL research (parsing, tagging, ASR, etc)
}

\todo[inline,caption={Call to action}]{
\cite{pedersen2010}

CL Research can and should have an extremely low barrier to entry. Allow for lone researchers and small groups the opportunity to participate.
}

\section{Related Work}

\todo[inline,caption={Open code examples}]{
Moses \cite{Moses}, Joshua \cite{Joshua-WMT}, SAMT \cite{samt2006}, Cunei \cite{Phillips2009}
}

\todo[inline,caption={Open data examples}]{
Europarl \cite{Koehn-europarl}, JRC-Acquis \cite{Steinberger-2006}, Fr-En $10^9$ \cite{WMT09-Findings}, News Commentary \cite{Callison-Burch2007a,Callison-Burch2008a}
}

\todo[inline,caption={Open workflow examples}]{
\cite{LoonyBin} and \cite{experiment.perl}
}

\section{Framework}

\missingfigure{Graph of dependencies and steps}

GNU Make \cite{gnumake}

\todo[inline,color=yellow,caption={parallel make}]{
Many tasks, such as preprocessing numerous training files, are not dependent on one another. In such cases {\tt make} can be configured to execute multiple processes simultaneously on a single multi-processor machine. In cases where scheduled distributed computing environments such as the Sun Grid Engine are configured, well-designed make files can be run by make variants ({\tt distmake}, SGE {\tt qmake}, Sun Studio {\tt dmake}) which distribute tasks to multiple machines using the distributed scheduler.
}

\section{Reproducible Results}


\bibliographystyle{acl}
\bibliography{bibliography}

\end{document}
