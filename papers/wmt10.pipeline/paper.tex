\documentclass[11pt]{article}
%\usepackage[%
%  colorlinks,%
%  linkcolor=black,%
%  anchorcolor=black,%
%  citecolor=black,%
%  filecolor=black,%
%  menucolor=black,%
%  pagecolor=black,%
%  urlcolor=black,%
%  %hyperindex=false,%
%  bookmarksopen=true,%
%  pdfusetitle,%
%  pdfpagelabels,
%  naturalnames
%]{hyperref}
\usepackage{times}
\usepackage{url}
\usepackage{latexsym}

%\usepackage[colorlinks]{hyperref}

\usepackage{tikz}
\usetikzlibrary{fit,shapes.misc,shapes}

\usepackage{acl2010}
\usepackage[colorinlistoftodos]{todonotes}


%\title{Empirical Questions in Machine Translation\
 %--- Reproducible Results for the JHU WMT10 System}
\title{Reproducible Results in Parsing-Based Machine Translation:\\
The JHU Shared Task Submission}

\author{Lane Schwartz\
\thanks{Research conducted as a visiting researcher at Johns Hopkins University}\
\\University of Minnesota\\Minneapolis, MN\\
{\tt lane@cs.umn.edu} \And
Chris Callison-Burch \\
Johns Hopkins University \\
Baltimore, MD\\
{\tt ccb@cs.jhu.edu}}

\date{}

\begin{document}
\maketitle

\begin{abstract}
\end{abstract}

%\listoftodos

\section{Introduction}



\todo[inline,caption={Motivation}]{
Papers are not reproducible.
\begin{itemize}
\item Don't list code
\item Don't provide data
\item Don't publish scripts to reproduce
\end{itemize}

MT is complex, much more so than other CL research (parsing, tagging, ASR, etc)
}

\todo[inline,caption={Call to action}]{
\cite{pedersen2010}

CL Research can and should have an extremely low barrier to entry. Allow for lone researchers and small groups the opportunity to participate.
}

\section{Related Work}

\todo[inline,caption={Open code examples}]{
Moses \cite{Moses}, Joshua \cite{Joshua-WMT}, SAMT \cite{samt2006}, Cunei \cite{Phillips2009}
}

\todo[inline,caption={Open data examples}]{
Europarl \cite{Koehn-europarl}, JRC-Acquis \cite{Steinberger-2006}, Fr-En $10^9$ \cite{WMT09-Findings}, News Commentary \cite{Callison-Burch2007a,Callison-Burch2008a}
}

\todo[inline,caption={Open workflow examples}]{
\cite{LoonyBin} and \cite{experiment.perl}
}

\todo[inline,caption={Parsing-based SMT}] {
Parsing-based SMT: \cite{SCALE-report} Hiero grammar extraction \cite{Lopez2008}
}

\section{Framework}

%\missingfigure{Graph of dependencies and steps}
\begin{figure*}
%\include{mt-workflow-preprocess}
\include{figure.pgf}
%\begin{tikzpicture}[shape=rounded rectangle]
%\node (PreprocessLabel) {\underline{\bf Preprocess Data}};
%\node (CompressedData) [style=draw,below of= PreprocessLabel]  {Compressed Data};
%\node (DecompressedData) [style=draw,below of=CompressedData] {Decompressed Data};a
%\node (RemoveXML) [style=draw,below of=DecompressedData,text width=28mm]  {Test Set Data Stripped of XML};
%\node (Tokenize) [style=draw,below of= RemoveXML]  {Tokenized Data};
%\node (Normalize) [style=draw,below of=Tokenize] {Normalized Data};
%\node [style=draw,shape=rectangle,fit=(PreprocessLabel) (CompressedData) (DecompressedData) (RemoveXML) (Tokenize) (Normalize)] {};
%
%\draw [->] (CompressedData) -- (DecompressedData);
%\draw [->] (DecompressedData) -- (RemoveXML);
%\draw [->] (RemoveXML) -- (Tokenize);
%\draw [->] (Tokenize) -- (Normalize);
%\draw [->] (DecompressedData) -- (Tokenize);
%\end{tikzpicture}

\caption{Machine translation workflow}
\end{figure*}

GNU Make \cite{gnumake}

\todo[inline,color=yellow,caption={parallel make}]{
Many tasks, such as preprocessing numerous training files, are not dependent on one another. In such cases {\tt make} can be configured to execute multiple processes simultaneously on a single multi-processor machine. In cases where scheduled distributed computing environments such as the Sun Grid Engine are configured, make files can be processed by scheduler-aware {\tt make} variants ({\tt distmake}, SGE {\tt qmake}, Sun Studio {\tt dmake}) which distribute outstanding tasks to available distributed machines using the relevant distributed scheduler.
}

\section{Reproducible Results}

\begin{table}[h]
\begin{center}
\begin{tabular}{|l|l||c|c|c|}
\hline
\bf Source & \bf Target & \bf BLEU & \bf BLEU- & \bf TER \\
& & & \bf cased & \\
\hline
German & English & 21.3 & 19.5 & 0.660 \\ \hline
English & German & 15.2  & 14.6  & 0.738 \\ \hline
French & English & 27.7 & 26.4 & 0.614 \\ \hline
English & French & 23.8 & 22.8 & 0.681 \\ \hline
Spanish & English & 29.0 & 27.6 & 0.595 \\ \hline
English & Spanish & 28.1 & 26.5 & 0.596  \\ \hline
\end{tabular}
\end{center}
\caption{\label{scores} Automatic metric scores for the test set newstest2010 }
\end{table}


\begin{table}[h]
\begin{center}
\begin{tabular}{|l|l||c|}
\hline
\bf Source & \bf Target & \bf BLEU  \\
\hline
German & English & 18.19  \\ \hline
English & German & 13.57 \\ \hline
French & English & 26.41 \\ \hline
English & French & 25.28 \\ \hline
Spanish & English & 25.28 \\ \hline
English & Spanish & 24.02  \\ \hline
\end{tabular}
\end{center}
\caption{\label{devtest-scores} Automatic metric scores for the development test set newstest2009}
\end{table}

\todo[inline,caption={URL}]{
List how to download: \url{http://sourceforge.net/projects/joshua/files/joshua/1.3/wmt2010-experiment.tgz}
}

\section*{Acknowledgements}
This work was supported by the DARPA GALE program (Contract No HR0011-06-2-0001).

\bibliographystyle{acl}
\bibliography{bibliography}

\end{document}
