%% Sample file $Id: PBML_note.tex 92 2007-12-18 22:59:46Z zw $
% 
% This is a sample file for the PBML note.
% 
% The Prague Bulletin of Mathematical Linguistics is typeset by XeLaTeX. This
% means that it is not guaranteed that the line and page breaks will come up
% exactly the same in the final printed version as in your computer. It is
% not required that you use XeLaTeX and the same fonts as will be used in the
% journal but it is better to do so if you could. You can even used standard
% LaTeX if XeLaTeX is not available on your computer.
% 
% 
% Document class
% ==============
% 
% The Prague Bulletin of Mathematical Linguistics uses its own document class.
% You will thus start your document with:

\documentclass{pbml}

% If you process the document with XeLaTeX, fonts Minion Pro (distributed
% with Adobe Reader) and DejaVu
% (http://dejavu.sourceforge.net/wiki/index.php/Main_Page). If you do not
% have these fonts and are not able to install them, you can instruct XeLaTeX
% to use its default fonts by issuing \document[nofonts]{pbml}.
% 
% 
% Packages
% ========
% 
% The note should not require any package.
% 
% 
% Beginning of the document
% =========================
% 
% Use the standard command:

\begin{document}


% Note title and author
% =====================
% 
% Due to journal organization the note title and author must be specified
% AFTER \begin{document}. Give the information as:

\PBMLnote{title={Pig, Pepper, Baby}, author=Alice}

% The braces are mandatory only if the value contains a comma as in the first
% author's address but they never make any harm, you can safely put all
% values into braces.
% 
% 
% The body of the note
% ====================
% 
% The notes usually have no sectioning but you can use sections, tables and
% figures similarly as in an article. See PBML_article.tex for details.

% End of the note
% ===============
% 
% The note must end with:

\end{document}
